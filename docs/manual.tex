\documentclass{article}
\title{X16 Edit user manual}
\date{\today}
\usepackage{hyperref}
\usepackage{float}
\usepackage{longtable}
\usepackage{graphicx}
\usepackage{wrapfig}

\begin{document}

\maketitle

\section{Introduction}

    X16 Edit is a simple text editor written in 65C02 assembly especially
    for the Commander X16 retro computer. 
    
    The look and feel of the program is inspired by GNU Nano, but there are,
    naturally, many differences.
    
    One primary design goal is to support editing of large text 
    files. A lot of care has been put into the design of
    the internal memory model to make this possible.

    The Commander X16 was devised by David Murray a.k.a. the 
    8-Bit Guy. For more information on the platform, go to 
    \href{http://www.commanderx16.com}{www.commanderx16.com}.

\section{Getting help}

    X16 Edit is controlled by keyboard shortcuts. The most frequently used 
    shortcuts are always displayed at the bottom of the screen.
    
    There is also a built-in help screen, which is displayed when you press
    Ctrl+G. The help screen lists all keyboard shortcuts with a short description
    of what they do.

    All keyboard shortcuts are also listed in section \ref{commands} of this manual.

\section{Basic usage}

    \subsection{Entering text}
           
        X16 Edit is a modeless editor. As soon as it is started, everything you
        type on the keyboard is inserted into the text buffer.
        
        By default, line breaks are not made automatically. There is no limit
        to the length of a line, other than the available memory. If 
        the current line does not fit on the screen, it is 
        scrolled horizontally. 
        
        There is, however, an optional automatic 
        word wrap mode. Read more about that in section \ref{wordwrap}.
       
    \subsection{Moving the cursor}

        The cursor is primarily moved by the standard arrow keys.

        You may move to start of line with the Home key, and to end of line with the End key or Shift+Home.

        The PgUp and PgDn keys are supported. Alternatively you may press
        Ctrl+Y for page up or Ctrl+V for page down.

        Finally, it is also possible to move the cursor to a specific line number 
        with the go to line feature (Ctrl+L).

        Both the Backspace and the Delete keys have their standard meaning.

    \subsection{Commands}
    \label{commands}
    
        Commands are selected with keyboard shortcuts. They may be entered in the following three ways:

        \begin{itemize}

            \item Press and hold down the Ctrl key or the left Alt key at the same time as you press a command key. This is the
            preferred way of selecting commands.
            
            \item Alternatively you may press and release the Esc key. The program is now ready to receive
            a command key, and a message stating this is displayed in the status bar. 
            Press a command key, without holding down Ctrl, or Esc to abort. This option is mostly
            made as a backup, in case the Ctrl+key sequence does not work. 
            
            \item Finally, some commands are available via an optional function key.
        \end{itemize}

	\begin{longtable}{c c l}
		\caption{List of keyboard shortcuts} \\
	    \textbf{Ctrl or Esc} & \textbf{F-key} & \textbf{Description} \\
	    \hline \\
            G    & F1    & Display built-in help screen \\
            X    & F2    & Exit from the program \\
            O    & F3    & Write text buffer to file \\
            R    & F5    & Open and read a file into the buffer \\
            N    & ---   & Create new text buffer \\
            J    & F4    & Justify text buffer \\
            Y    & F7    & Page up \\
            V    & F8    & Page down \\
            K    & ---   & Cut current line and save it to clipboard \\
            C    & ---   & Copy current line to clipboard \\
            U    & ---   & Paste (uncut) all content from clipboard \\
            DEL  & ---   & Delets current line, no copy to clipboard \\
            W    & F6    & Search and find string in buffer (case sensitive) \\
            S    & ---   & Replace string (case sensitive) \\
            L    & ---   & Goto line number \\
            A    & ---   & Toggle auto indent on/off \\
            Z    & ---   & Toggle word wrap on/off \\
            E    & ---   & Change charset\\
            I    & ---   & Invoke DOS command\\
            D    & ---   & Set file storage device number, default is 8 \\
            T    &  ---  & Cycle through text colors \\
            B    &  ---  & Cycle through background colors \\
            M    &  ---  & Show memory usage (1 block=251 bytes) \\
            space& ---   & Insert non-breaking space 
        \end{longtable}

        \noindent The tab stop width is set by first pressing and releasing Esc and then one of the digits 1--9.

        The command key bindings are user configurable, see section \ref{keybindings}.

    \subsection{User interface}
	
	The user interface is inspired by GNU Nano, and should be mostly self-explanatory. It consists of the following main parts:
	
	\begin{itemize}
		\item The title bar
		\item The status bar
		\item The shortcut list
		\item The editor area
	\end{itemize}
	
	\noindent The \textit{title bar} is displayed on the first row of the screen. You find the program name and version to the left. 
	The current file name is shown at the center. If the text buffer has never been saved to file, the string "NEW BUFFER" is
	shown instead of a file name. At the right edge, the letters "MOD" are shown if the text buffer has been modified
	since last saved to file.
	
	The \textit{status bar} is the line third from the bottom of the screen. All messages from the program are displayed in
	the status bar. Press any key to hide a message. If the program needs to prompt you for input, the prompt is also shown in the status bar. 
	Press Enter to confirm input or Esc to abort the operation.
	
	The last two lines at the bottom of the screen contain the \textit{shortcut list}. The most frequently used commands are
	shown here.
	
	The \textit{editor area} covers all lines between the title bar and the status bar. 

    \begin{figure}[H]
        \caption{Main elements of the user interface, (1) title bar, (2) editor area, (3) status bar, and (4) shortcut list.}
        \centering\includegraphics[width=0.67\textwidth]{interface.png}
    \end{figure}
	
	\subsection{Non-breaking space}

        By default, the Commander X16 interprets Shift+space as a non-breaking space. 
        
        To prevent typing errors, the editor will, however, insert a normal space character even if the Shift key is held down.
        Non-breaking spaces are rarely used when you edit plain text files. And it is quite easy to type them by mistake,
        especially if the character immediately before or after the space requires the Shift key.

        If you actually want to insert a non-breaking space you may type Ctrl+space.

    \subsection{Text buffer size}

        X16 Edit stores its text buffer in banked RAM, which by default is 512 kB,
        expandable to 2 MB. 
        
        The program does not use virtual memory or otherwise
        switch memory to the file system. Consequently, it cannot handle text
        buffers larger than the available banked RAM.

        You may get the available free memory by pressing Ctrl+M. The available memory is 
        reported as number of blocks free. One block may hold at most 251 characters.

        If you have used all available memory, the editor will display a memory
        full message in the status bar. Further insertion of characters is ignored.

\section{Features}

    \subsection{Tab stops}

        The default tab stop width is four spaces. You may change the width by pressing and releasing the Esc key followed
        by one of the digits 1--9. The selected digit indicates the tab stop width.
        
        The tab key works by inserting blank spaces until reaching the next tab stop.

    \subsection{Auto indent}
    
        Use the auto indent feature to keep the level of indentation when line breaks are inserted manually or automatically by
        the word wrap feature.

        Auto indent is turned off when the editor starts. To toggle the feature on or off, press Ctrl+A.
    
    \subsection{Word wrap and text justification}
    \label{wordwrap}

        By default, automatic word wrap is turned off. If you type a line longer than the width of the screen, the line
        is scrolled horizontally.

        Automatic word wrap is toggled on or off with Ctrl+Z. When turned on, you are prompted for the column where
        to wrap. The feature works in a simplified way. When you reach the right margin, the editor breaks the
        line after the previous blank space. If there is no blank space on the line, the line break is inserted
        at the right margin. If you delete characters from a line or if you insert characters at the beginning of a line,
        the line breaks are not recalculated.

        There is, however, a command to justify the whole text buffer (Ctrl+J). The justify command breaks the text buffer into
        paragraphs, and recalculates the word wrap using the line length set by the the word wrap function (default 80). The
        word wrap function need not be enabled to run the justify function.

        The justify command interprets the start of a new paragraph the same way as GNU Nano. 
        
        When auto indent is turned off a new paragraph is considered to begin

        \begin{itemize}
		    \item if two or more consequtive line breaks are found, or
		    \item if a line starts with one or more blank spaces.
	    \end{itemize}

        If auto indent is turned on, a new paragraph is considered to begin

        \begin{itemize}
		    \item if two or more consequtive line breaks are found, 
            \item if a line contains only blank space characters, or
		    \item if the level of indent is different from the previous line.
	    \end{itemize}      

    \subsection{Cut, copy and paste}

        X16 Edit supports the traditional cut, copy, and paste features.

        The copy (Ctrl+P) and cut (Ctrl+K) commands copy all of the current line to the clipboard. It is not possible to select 
        a part of a line. The lines you copy or cut are placed in the clipboard in the order they were copied or cut.
        
        The clipboard may hold a maximum of 3 kB of data. If you reach that limit, the program will inform you.

        Pasting or uncutting (Ctrl+U) will insert all content in the clipboard at the position of the cursor. The clipboard
        is cleared upon the first cut or copy after the clipboard content was pasted into the document, which is the
        same behavior as GNU Nano.

    \subsection{Search and replace}

        X16 Edit also supports search (Ctrl+W) and replace (Ctrl+S).

        Both search and replace are case sensitive. Search starts from the
        cursor position and is only forward looking.

        When searching for a string, the editor moves the cursor to the start
        of the next occurrence. If the string is not found a message is
        displayed in the status bar.

        When replacing a string, you are given the option to only replace the
        next occurrence or all subsequent occurrences.

    \subsection{Supported character sets}

        X16 Edit supports the three character sets of the Commander X16:

        \begin{itemize}
            \item PETSCII upper case/graphics. This is the default mode of both the Commander X16 and the C64.

            \item PETSCII upper/lower case. This is the same mode as is available on the C64.

            \item ISO character set. This mode is new, and there is no corresponding mode supported by 
            Commodore 8 bit computers. Text is encoded according to ISO-8859-15, making it
            easier to transfer files to and from modern computers.
        \end{itemize}

        \noindent On startup, X16 Edit detects the current character set. If the detection is successful, it
        continues using that character set.

        If the current character set for some reason cannot be detected, the program defaults to ISO mode.

        During the operation of the program, it is possible to change the character set. Press Ctrl+E to cycle
        through the options.

    \subsection{Line break encoding}

        The selected character set mode determines how the editor encodes line breaks when writing the
        text buffer to file. 
        
        In both PETSCII modes, it uses a single CR to indicate line breaks. This
        is the closest you get to a standard for Commodore 8 bit computers. This is also the
        setup most likely to work with applications written for Commodore computers.
        
        In ISO mode it uses
        a single LF to mark line breaks. This is the standard used today by Linux and MacOS. Following
        this standard makes it easier to transfer text files to or from modern computers.

        Internally the editor uses a single LF as line break marker. On reading a file it converts
        all occurrences of CR to LF. When writing the text buffer back to file, the line breaks
        are converted to CR if in PETSCII mode.

        In the event you want to save a PETSCII file with LF line breaks or an ISO file with
        CR line breaks, all is not lost. Use the preferred character set mode while
        editing the file. Before saving, switch to ISO mode for LF line breaks or PETSCII mode
        for CR line breaks.

    \subsection{Background and text color}

        Both the background and the text color may be changed while using the editor. The program
        runs in 16 color text mode, so there are 16 background and 16 text colors to choose from.

        Press Ctrl+T to cycle through background color options.
        Press Ctrl+B to cycle through text color options.

        \begin{figure}[H]
            \caption{Use your favorite colors!}
            \centering\includegraphics[width=0.5\textwidth]{colors.png}
        \end{figure}
       
\section{File handling}

    \subsection{Reading and writing files}

    X16 Edit file handling is centered around functions for reading a file into the text buffer and writing
    the text buffer to a file.

    Press Ctrl+R to read from a file or Ctrl+O to write to a file. The file name may 
    be typed in at the prompt that is displayed. A file name may 
    be just the name of the file or a complete CBM DOS path.

    If you are in ISO mode, it's recommended to type file names in upper case. Otherwise file names will not
    be readable when listed in PETSCII upper case/graphics mode.


    \subsection{File browser}

    At the prompt where a file name is typed in you may alternatively press Ctrl+T to show the
    built-in file browser. The file browser lists the content of the current directory. To select a file,
    first highlight it with the up or down arrow keys, and then press Enter.

    If the selected item is a directory, it will be made the new current directory, and its content will
    be displayed in the file browser.

    The file browser shows at most 50 files or directories on one page. If not all items fit on one page 
    the listing is ended with "--- MORE ---". In case the items are spread over several pages, you may go to 
    the next page with Ctrl+V and back to a previous page with Ctrl+Y.

    If there are no more items to show the listing is ended with "--- END ---".


    \subsection{Disk commands}

    X16 Edit lets you send commands to the disk. To invoke a command, press Ctrl+I, and then enter
    the command at the prompt.

    Any valid command may be sent. You find a list of commands at 
    \href{https://github.com/commanderx16/x16-rom/tree/master/dos}{X16 DOS}. Some of the most useful commands are:

     \begin{itemize} 
            \item "C:dst=src", copy src file to dst file
            \item "R:dst=src", rename src file to dst file
            \item "S:filename", delete filename
            \item "CD:dirname", change current directory
            \item "RD:dirname", remove directory
    \end{itemize}
    
    Please be careful though. There is nothing stopping you from deleting files or even formatting the
    disk.


    \subsection{Change device number}

    By default the file handling functions use device \#8. The device
    number may be changed by pressing Ctrl+D.

\section{User-configurable key bindings}
\label{keybindings}

The shortcut key bindings used in X16 Edit are user-configurable.

On startup, the program reads custom key bindings from the file X16EDITRC in the
root folder of the SD card. If that file is not available, the default key bindings are
used.

The X16EDITRC file has a very simple format. It is only a stream of bytes, one for every
shortcut, with no metadata.

Each key is represented by the value returned by the Kernal function GETIN when pressed
without Shift, Ctrl, Alt, or any other modifier key held down.

The extra keys suppurted by the editor are represented by the following values:

\begin{itemize}
    \item \$15 = Delete
    \item \$16 = End
    \item \$17 = PgUp
    \item \$18 = PgDn
    \item \$1a = Insert 
\end{itemize}

The values in X16EDITRC are bound to shortcuts in a fixed order, the same order as
the shortcuts appear in section \ref{commands}.

If X16EDITRC holds fewer values than there are shortcuts, the editor will use
default bindings for the remaining ones. If there are more values in the file than there
are bindings, the excess is ignored.

To bind the first three shortcuts -- Show help, Exit, and Write Out -- to other keys,
the file could for example begin with the following three bytes:

\begin{verbatim}
    $72, $81, $83 ;ASCII values of H, Q, and S
\end{verbatim}

To make it a bit easier to setup X16EDITRC, you may use the provided tool for
this purpuse (X16EDIT-KEYBINDINGS.PRG).

\section{Application Interface}

    \subsection{Program Entry points}

        X16 Edit has a small Application Interface (API) that consists of the three
        entry points described in more detail below.

        The entry points may be used by other programs to start X16 Edit, and to
        control how it's configured on startup.

    \subsection{Default entry point}

        The default entry point starts the editor with default options and with an
        empty new text buffer.

        The call address is \$080D in the RAM version and \$C000
        in the ROM version of the program. The default entry point is called in
        the RAM version when started with the RUN command.

        Before calling the entry point in the ROM version of the program, you need to set the following
        parameters.

        \begin{longtable}{l l}
            \caption{List of parameters} \\
	        \textbf{Register} & \textbf{Description} \\
	        \hline \\
            X  & First RAM bank used by the program \\
            Y  & Last RAM bank used by the program \\
        \end{longtable}

        \noindent Those parameters are ignored in the RAM version of the program, and
        defaults to using all available banked RAM, banks 1--255.

        The purpose of setting the first and last RAM banks, is to reserve parts of banked RAM
        for other purposes.

    \subsection{Load file entry point}

        On startup, the load file entry point loads a specified text file from the SD Card.

        The call address is \$0810 in the RAM version and \$C003 in the ROM version of the program.
        Before calling the entry point, you need to set the following parameters.

        \begin{longtable}{l l l}
            \caption{List of parameters} \\
	        \textbf{Register} & \textbf{Address} & \textbf{Description} \\
	        \hline \\
            X   &          & First RAM bank used by the program \\
            Y   &          & Last RAM bank used by the program \\
            r0  & \$02--03 & Pointer to file name \\
            r1L & \$04     & File name length, or 0=no file \\
        \end{longtable}

        \noindent If the specified file does not exist, the editor will display display an
        error message. If the file name length is 0, the program will not try
        to load any text file on startup.

        The first and last RAM bank settings control what parts of banked RAM is
        used by the program. This option may be used to reserve parts of banked RAM
        for other purposes.

    \subsection{Load file with options entry point}

        On startup, the load file with options entry point applies the specified editor options, and then loads
        a text file from the SD Card.

        The call address is \$0813 in the RAM version and \$C006 in the ROM version of the program.
        Before calling the entry point, you need to set the following parameters.

        \begin{longtable}{l l l l}
            \caption{List of parameters} \\
	        \textbf{Register} & \textbf{Address} & \textbf{Bits} & \textbf{Description} \\
	        \hline \\
            X   &          &      & First RAM bank used by the program \\
            Y   &          &      & Last RAM bank used by the program \\
            r0  & \$02--03 &      & Pointer to file name \\
            r1L & \$04     &      & File name length, or 0=no file \\
            r1H & \$05     & 0    & Auto indent on/off \\
            r1H & \$05     & 1    & Word wrap on/off \\
            r1H & \$05     & 2--7 & Unused \\
            r2L & \$06     &      & Tab width (1..9) \\
            r2H & \$07     &      & Word wrap position (10..250) \\
            r3L & \$08     &      & Current device number (8..30) \\
            r3H & \$09     & 0--3 & Text color \\
            r3H & \$09     & 4--7 & Background color \\
            r4L & \$0A     & 0--3 & Header text color \\
            r4L & \$0A     & 4--7 & Header background color \\
            r4H & \$0B     & 0--3 & Status bar text color \\
            r4H & \$0B     & 4--7 & Status bar background color \\
        \end{longtable}

        \noindent Parameters out of range are ignored silently, and the default
        values are used instead.

        Color settings are ignored if both the text and background color is 0
        ("black on black").

        If the specified file does not exist, the editor will display an
        error message. If the file name length is 0, the program will not try
        to load any text file on startup.

        The first and last RAM bank settings control what parts of banked RAM is
        used by the program. This option may be used to reserve parts of banked RAM
        for other purposes.

    \subsection{Code samples}

        Below are some code samples showing how to use the program entry points. The
        samples are made to work with the RAM version of the program. More information
        on how to use the entry points in the ROM version is found in the ROM notes
        document.

        First, code that invokes the load file entry point.

\begin{verbatim}
    ldx #1      ;First RAM bank used by the editor
    ldy #255    ;Last RAM bank used by the editor
    lda #<fname ;Pointer to file name, LSB
    sta $02     ;r0L
    lda #>fname ;Pointer to file name, MSB
    sta $03     ;r0H
    lda #4      ;File name length
    sta $04     ;r1L
    jsr $080A   ;Load file entry point
    rts
    fname: .byt "test"
\end{verbatim}

        Secondly, code that invokes the load file with options entry point.

\begin{verbatim}
    ldx #1          ;First RAM bank used by the editor
    ldy #255        ;Last RAM bank used by the editor
    lda #<fname     ;Pointer to file name, LSB
    sta $02         ;r0L
    lda #>fname     ;Pointer to file name, MSB
    sta $03         ;r0H
    lda #4          ;File name length
    sta $04         ;r1L
    lda #1
    sta $05         ;r1H, auto indent on, word wrap off
    lda #4
    sta $06         ;r2L, tab width
    lda #80
    sta $07         ;r2H, word wrap position
    lda #8
    sta $08         ;r3L, device number
    lda #1|11<<4    ;Text white, background light green
    sta $09         ;r3H, screen color
    lda #7|0<<4     ;Text yellow, background black
    sta $0A         ;r4L, header color
    lda #0          ;Ignore, use default color
    sta $0B         ;r4H, status bar color
    jsr $0813
    rts
    fname: .byt "test"
\end{verbatim}

\section{License}

	Copyright 2020--2021, Stefan Jakobsson.

	The X16 Edit program, including this manual, is released under the GNU General Public License version 3 or later.
    The program is free software and comes with ABSOLUTELY NO WARRANTY. You may redistribute and/or modify it under the 
    terms of the GNU General Public License as pub­lished by the Free Software Foundation, either version 3 of the License, 
    or, at your option, any later version. For detailed terms see license file distributed with the program. 
    Also available from \href{https://www.gnu.org/licenses}{www.gnu.org/licenses}.

\end{document}