\documentclass{article}
\title{X16 Edit user manual}
\date{\today}

\setlength{\parindent}{0pt}
\setlength{\parskip}{1.2ex}

\usepackage{hyperref}
\renewcommand{\familydefault}{\sfdefault}
\usepackage{graphicx}
\usepackage{wrapfig}
\usepackage{float}
\usepackage{wrapfig}
\usepackage{longtable}

\begin{document}

\begin{huge}
    X16 Edit user manual
\end{huge}

\vspace{1em}
\today

\vspace{4em}
\tableofcontents
\vspace{4em}

\section{Introduction}

Thank you for trying out X16 Edit!

X16 Edit is a simple text editor written in 65C02 assembly especially
for the Commander X16 retro computer. 

The look and feel of the program is inspired by GNU Nano. There are,
naturally, a lot of differences, but you will feel at home if you are
used to Nano.

The Commander X16 was devised by David Murray a.k.a. the 
8-Bit Guy. For more information on the platform, go to 
\href{http://www.commanderx16.com}{www.commanderx16.com}.

\section{Basic usage}
 
    \subsection{Getting started}
        The default version of X16 Edit consists of just one file, 
        X16EDIT-x.x.x.PRG, where x.x.x is the version number.
 
        There are several more executable files in the X16 Edit project, but you may
        ignore them for the time being.
 
        The editor is loaded and run in the same way as a BASIC program.
        
        To run the editor on hardware, just store the program file
        on an SD card. Insert the SD card into your X16 computer. Type
        \texttt{LOAD"X16EDIT-x.x.x.PRG"} to load the program. And then \texttt{RUN} to start it.
 
        If you want to run the editor in the x16emu emulator, you first need to 
        store the X16 Edit executable in the host computer's file system.
        Then type the following command in the terminal:
 
\begin{verbatim}
    x16emu -sdcard sdcard.img -prg X16EDIT-x.x.x.PRG -run
\end{verbatim}
        An SD card image needs to be attached to the
        emulator with the "-sdcard" option for the editor to
        fully work. When you download the emulator you get an empty SD card
        image (sdcard.img) that can be used for this purpose.
 
        A valid path to the editor executable must be specified after
        the "-prg" option. Otherwise the emulator will not know where
        to find the program.
 
        The "-run" option automatically starts the editor after the emulator
        is launched. You may remove this option and start the editor manually
        by typing \texttt{RUN} in the emulator.
 
    \subsection{Entering text}
        X16 Edit is a modeless text editor. As soon as it is started, everything
        you type is stored in the text buffer.
 
        The cursor is moved with keys normally used for that purpose on a modern
        keyboard, \textit{i.e.} the arrow keys, Tab, Home, End, PgUp and PgDn.

        The go to line feature (Ctrl+L) lets you move the cursor to a specific line number.
 
        As you type and reach the right margin the editor does not insert a line break
        by default. Instead the current line is scrolled horizontally to
        make room for more characters. There is no limit to the length of a line
        other than the available memory.
 
    \subsection{Saving and loading text files}
        The current text buffer is saved to file on the SD card when you press Ctrl+O. Type in the 
        file name you like to use, and press the ENTER key. You will
        be prompted to confirm before overwriting av file that already exists.
         
        To load an already existing file from the SD card, press Ctrl+R. 
        Just type in the name of the file you want to load, and then press the ENTER key.
        
    \subsection{Keyboard shortcuts}
        X16 Edit is controlled by keyboard shortcuts. You find a complete list
        of shortcuts in Appendix A.

        Shortcuts are primarily selected by pressing and holding down Ctrl at the
        same time as you press a key that is linked to the wanted command. Instead of
        the Ctrl key, you may hold down the Left Alt key.

        It is also possible to press and release the ESC key. A message is displayed
        in the status bar indicating that the program is ready to receive a command.
        Just press the wanted command key without holding down any modifier key.
      
    \subsection{User interface}
        X16 Edit's user interface is similar to GNU Nano, and should be mostly self-explanatory.
        It consists of the following main parts:

        \begin{itemize}
            \item Title bar
            \item Editor area
            \item Status bar
            \item Shortcut list
            \item Cursor position display
        \end{itemize}

        The \textit{title bar} is at the top row of the screen. The name of the current file
        is displayed at the center. If the text buffer hasn not been saved to file, it displays
        "New text buffer". At the right edge the letters "MOD" are shown if the
        text buffer has been changed since last saved to file.

        The \textit{editor area} is right below the title bar. It takes up most of the screen, and
        it is here you do all text editing.

        The \textit{status bar} is at the third row from the bottom of the screen. All messages
        from the program are displayed in the status bar. If the program needs to prompt you
        for input, the input prompt is displayed here as well. 
        
        At the input prompt you may either press ENTER to confirm your input or press ESC 
        to abort the current operation.

        The \textit{shortcut list} takes up the last two rows of the screen. Here you find
        the most common keyboard shortcuts. 
        
        The \textit{cursor position} (row:column) is always displayed at the bottom right corner
        of the screen.

    \subsection{Built-in help}
        A help screen that lists the available keyboard shortcuts and a short description 
        of what each command do is displayed in the built-in help screen (Ctrl+G). 
    
\section{More on text editing}

    \subsection{Supported character sets}
        The editor supports the following built-in character sets:

        \begin{itemize}
            \item PETSCII upper case/graphics. This is the default mode of both the Commander X16 and the C64.

            \item PETSCII upper/lower case. This is the same mode as is available on the C64.

            \item ISO character set. This mode is new, and there is no corresponding mode supported by 
            Commodore 8 bit computers. Text is encoded according to ISO-8859-15, making it
            easier to transfer files to and from modern computers.
        \end{itemize}

        On startup, the editor detects the current character set and continues using that.

        You may change the character set by pressing Ctrl+E, which cycles through the options.

    \subsection{Tab stops}
        The default tab stop width is four spaces. 
        
        You may change the width by pressing and releasing the ESC key followed
        by one of the digits 1--9.
                
    \subsection{Auto-indent}
        The auto-indent feature is used to keep the level of indentation when line breaks are inserted manually or automatically by
        the word wrap feature.

        Auto-indent is turned off when the editor starts. To toggle the feature on or off, press Ctrl+A.

    \subsection{Word wrap}
        The editor does not do automatic line breaks by default. Instead the current line is scrolled horizontally
        when you reach the right margin.

        Automatic word wrap may be toggled on or off by pressing Ctrl+Z. When turned on you are prompted for the column
        where to wrap.

        The word wrap feature works in a very simplified way. When you reach the right margin the editor breaks the
        line after the previous blank space. If there is no blank space on the line, the line break is inserted
        at the right margin. The line break position is not recalculated if you delete characters from the line
        or if you insert new characters at the beginning of the line.

    \subsection{Text justification}
        The justify command (Ctrl+J) divides the text buffer into paragraphs and recalculates all line breaks
        according to the line width specified in the word wrap feature (default 80). The word wrap function 
        need not be enabled to run the justify function.
          
        When auto-indent is turned off a new paragraph is considered to begin

        \begin{itemize}
            \item if two or more consecutive line breaks are found, or
            \item if a line starts with one or more blank spaces.
        \end{itemize}

        If auto-indent is turned on, a new paragraph is considered to begin

        \begin{itemize}
            \item if two or more consecutive line breaks are found, 
            \item if a line contains only blank space characters, or
            \item if the level of indent is different from the previous line.
        \end{itemize}

    \subsection{Cut, copy and paste}
        The editor supports cut, copy, and paste.

        The copy (Ctrl+C) and cut (Ctrl+K) commands copy a whole line to the clipboard. It is not possible to select 
        just a part of a line.
        
        The clipboard may hold a maximum of 3 kB of data.

        Uncut (Ctrl+U) pastes all content in the clipboard at the cursor position. It is possible
        to do repeated pastes. The clipboard is cleared upon the first cut or copy after uncutting.

    \subsection{Search and replace}
        The search command (Ctrl+W) lets you search the text buffer for
        the occurence of a particular string. The replace function (Ctrl+S)
        lets you replace it with another string.

        Both search and replace are case sensitive. Search starts from the
        cursor position and is forward looking.

        If the string you are searching for is found, the editor moves the cursor to the start of the
        search match.

        When replacing a string, you are given the option to only replace the
        next occurrence or all subsequent occurrences.

\section{More on file handling}

    \subsection{The built-in file browser}
        At the prompt where a file name is typed in you may alternatively press Ctrl+T to show the
        built-in file browser. The file browser is used to move between directories and select
        files.
        
        To select an item in the file browser, first highlight it with the up or down arrow keys, and then press Enter.

        If the selected item is a directory, it will be made the new current directory, and its content will
        be displayed in the file browser.

        The file browser shows at most 50 files or directories on one page. If not all items fit on one page 
        the listing is ended with "--- MORE ---". In case the items are spread over several pages, you may go to 
        the next page with PgUp or Ctrl+V and back to a previous page with PgDn or Ctrl+Y.

        If there are no more items to show the listing is ended with "--- END ---".

    \subsection{Change disk drive device number}
        By default the file handling functions use device \#8. The device
        number may be changed by pressing Ctrl+D.

    \subsection{Disk drive commands}
        You may invoke disk drive commands by pressing Ctrl+I. The raw disk drive command
        is entered in the prompt that is displayed.

        Any valid command may be invoked. Some of the most useful commands are:

     \begin{itemize} 
            \item "C:dst=src", copy src file to dst file
            \item "R:dst=src", rename src file to dst file
            \item "S:filename", delete filename
            \item "CD:dirname", change current directory
            \item "MD:dirname", create directory
            \item "RD:dirname", remove directory
    \end{itemize}
    
        Just be careful! There is nothing stopping you from deleting files or even formatting the
        disk.

\section{Miscellaneous functions}

    \subsection{User-configurable keybindings}
        The shortcut key bindings in X16 Edit are user-configuarable.

        On startup the editor reads custom key bindings from the file X16EDITRC in the root directory
        of the SD card. If that file is not found, the default key bindings are
        used.
        
        X16EDITRC is simply a stream of bytes representing each custom shortcut without any metadata. 
        Each key is represented by the value returned by the Kernal function GETIN when pressed
        without any other modifier key.
        
        The extra keys supported by the editor are represented by the following values:
        
        \begin{itemize}
            \item \$15 = Delete
            \item \$16 = End
            \item \$17 = PgUp
            \item \$18 = PgDn
            \item \$1a = Insert 
        \end{itemize}
        
        The values in X16EDITRC are bound to shortcuts in a fixed order, the same order as
        the shortcuts appear in Appendix A.
        
        If X16EDITRC holds fewer values than there are shortcuts, the editor will use
        default bindings for the remaining ones. If there are more values in the file than there
        are shortcuts, the excess is ignored.
        
        To bind the first three shortcuts -- Show help, Exit, and Write Out -- to other keys,
        the file could for example begin with the following three bytes:
        
\begin{verbatim}
    $48, $51, $53 ;ASCII values of H, Q, and S
\end{verbatim}
        
        To make it a bit easier to setup X16EDITRC, you may use the provided tool
        (X16EDIT-KEYBINDINGS.PRG).

    \subsection{Line break encoding}

        X16 Edit internally uses LF (ASCII \$0A) as a line break marker.

        When reading a file, X16 Edit interprets every occurence of LF (ASCII \$0A) and CR (ASCII \$0C) as
        a line break.

        The selected character set determines how line breaks are encoded when saving the text buffer
        to file. If in one of the PETSCII modes, line breaks are encoded with CR. If in ISO mode, line
        breaks are encoded with LF.

    \subsection{Color settings}
        Both the background and the text color may be changed while using the editor. 
        The program runs in 16 color text mode, so there are 16 background and 16 text colors to choose from.

        Ctrl+T cycles through text color options.
        And Ctrl+B cycles through background colors.

    \subsection{Text buffer size}
        The text buffer is stored in banked RAM, which is 512 kB expandable to 2 MB.

        The text buffer may not exceed the available banked RAM.

        If you press Ctrl+M you get information on the remaining memory that can be
        used by the text buffer. It reports the 
        number of blocks free. A block may at most hold 251 characers.

\section{Advanced topics}

    \subsection{Building X16 Edit from source}
        If you like to build X16 Edit yourself you may download the source code from
        www.github.com/stefan-b-jakobsson/x16-edit.

        You will need the cc65 development tools to build the project.

        In the project's build folder, there are three build scripts:

        \begin{itemize}
            \item build-ram.sh, builds the RAM version
            \item build-hiram.sh, builds the high RAM version
            \item build-rom.sh, builds the ROM version
        \end{itemize}

    \subsection{X16 Edit HI version}
        The high RAM version is to be loaded into RAM at address \$6000. The purpose is to
        free up the start of BASIC RAM for other programs that can be loaded there at
        the same time.

        If present on an SD card, you may load the program with the command \texttt{LOAD"X16EDIT-HI-x.x.x.PRG",8,1}, and then
        start it with \texttt{SYS\$6000}.

        To run it in the emulator you may type the following command in the terminal:

\begin{verbatim}
    x16emu -sdcard sdcard.img -prg X16EDIT-HI-x.x.x.PRG,6000
\end{verbatim}
        In the emulator, type \texttt{"SYS\$6000"} to start the editor.

    \subsection{X16 Edit ROM version}
        The ROM version of the editor consists of one 16 kB ROM bank.

        To use the ROM version you first need to prepare a custom ROM image. When you
        download the emulator you get the standard rom.bin image. Append
        X16 Edit to the end of it with the following command (Linux and MacOS):

\begin{verbatim}
    cat rom.bin x16edit-rom-x.x.x.bin > customrom.bin
\end{verbatim}

        You need to write the custom ROM image to the actual ROM circuit to use it on
        real hardware.

        A custom ROM image can easily be attached to the emulator with
        the "-rom" option, for example as follows:

\begin{verbatim}
    x16emu -sdcard sdcard.img -rom customrom.bin
\end{verbatim}

        The editor's default entry point can be started from BASIC:

\begin{verbatim}
    BANK XX         ; where XX is the ROM bank number
    SYS $C000
\end{verbatim}

        If you like to use one of the other entry points,
        you must write a startup program that is stored in RAM. Code samples doing this
        are found in Appendix B.

    \subsection{The X16 Edit API}
        The X16 Edit API consists of the following three entry points:

        \begin{itemize}
            \item Default entry point, starts the editor with default options and an empty new text buffer
            \item Load file entry point, starts the editor with default options and then loads the specified text file
            \item Load file with options, start the editor with custom options and then loads the specified text file
        \end{itemize}

        Information on how to call the different entry points and code samples are available in Appendix B. 

\appendix
\newpage
\section{List of keyboard shortcuts}
    This is a complete list of keyboard shortcuts supported
    by X16 Edit. You may select commands in any of the following ways:

    \begin{itemize}
        \item Ctrl+key
        \item Left Alt+key
        \item Press and release ESC+key
        \item A function key from the F-key column
    \end{itemize}

    \begin{longtable}[l]{c c l}
        \textbf{Key} & \textbf{F-key} & \textbf{Description} \\
        \hline \\
        G    & F1    & Display built-in help screen \\
        X    & F2    & Exit from the program \\
        O    & F3    & Write text buffer to file \\
        R    & F5    & Open and read a file into the buffer \\
        N    & ---   & Create new text buffer \\
        J    & F4    & Justify text buffer \\
        Y    & F7    & Page up \\
        V    & F8    & Page down \\
        K    & ---   & Cut current line and save it to clipboard \\
        C    & ---   & Copy current line to clipboard \\
        U    & ---   & Paste (uncut) all content from clipboard \\
        DEL  & ---   & Deletes current line, no copy to clipboard \\
        W    & F6    & Search and find string in buffer (case sensitive) \\
        S    & ---   & Replace string (case sensitive) \\
        L    & ---   & Goto line number \\
        A    & ---   & Toggle auto indent on/off \\
        Z    & ---   & Toggle word wrap on/off \\
        E    & ---   & Change charset\\
        I    & ---   & Invoke DOS command\\
        D    & ---   & Set file storage device number, default is 8 \\
        T    &  ---  & Cycle through text colors \\
        B    &  ---  & Cycle through background colors \\
        M    &  ---  & Show memory usage (1 block=251 bytes) \\
        space& ---   & Insert non-breaking space 
    \end{longtable}

\newpage
\section{X16 Edit API}

    \subsection{Default entry point}

        \textbf{Purpose:} Start the editor with default options and an empty new text buffer

        \textbf{Call address:} \$080D (RAM version), \$6000 (HI version), \$C000 (ROM version)
    
        \textbf{Parameters:} None

    \subsection{Load file entry point}
        
        \textbf{Purpose:} Start the editor and load a specified text file

        \textbf{Call address:} \$0810 (RAM version), \$6003 (HI version), \$C003 (ROM version)

        \textbf{Parameters:} 

        \begin{longtable}[l]{l l l}
            \textbf{Register} & \textbf{Address} & \textbf{Description} \\
	        \hline \\
            X   &          & First RAM bank used by the program \\
            Y   &          & Last RAM bank used by the program \\
            r0  & \$02--03 & Pointer to file name \\
            r1L & \$04     & File name length, or 0=no file \\
        \end{longtable}

        If the specified file does not exist, the editor will display display an
        error message. If the file name length is 0, the program will not try
        to load any text file on startup.

        The first and last RAM bank settings control what parts of banked RAM is
        used by the program. This option may be used to reserve parts of banked RAM
        for other purposes.

    \subsection{Load file with options entry point}

        \textbf{Purpose:} Start editor with custom options and then load the specified text file

        \textbf{Call address:} \$0813 (RAM version), \$6006 (HI version), \$C006 (ROM version)

        \textbf{Parameters:}

        \begin{longtable}[l]{l l l l}
	        \textbf{Register} & \textbf{Address} & \textbf{Bits} & \textbf{Description} \\
	        \hline \\
            X   &          &      & First RAM bank used by the program \\
            Y   &          &      & Last RAM bank used by the program \\
            r0  & \$02--03 &      & Pointer to file name \\
            r1L & \$04     &      & File name length, or 0=no file \\
            r1H & \$05     & 0    & Auto indent on/off \\
            r1H & \$05     & 1    & Word wrap on/off \\
            r1H & \$05     & 2--7 & Unused \\
            r2L & \$06     &      & Tab width (1..9) \\
            r2H & \$07     &      & Word wrap position (10..250) \\
            r3L & \$08     &      & Current device number (8..30) \\
            r3H & \$09     & 0--3 & Text color \\
            r3H & \$09     & 4--7 & Background color \\
            r4L & \$0A     & 0--3 & Header text color \\
            r4L & \$0A     & 4--7 & Header background color \\
            r4H & \$0B     & 0--3 & Status bar text color \\
            r4H & \$0B     & 4--7 & Status bar background color \\
        \end{longtable}

        Parameters out of range are silently ignored, and default
        values are used instead.

        Color settings are ignored if both the text and background color is 0.

        If the specified file doesn't exist, the editor will display an
        error message. If the file name length is 0, the program will not try
        to load a text file.

        The first and last RAM bank settings control what parts of banked RAM is
        used by the program. This option may be used to reserve parts of banked RAM
        for other purposes.

    \subsection{Code samples for the RAM version}

    \textbf{Default entry point}
\begin{verbatim}
    jsr $080d  ; No parameters, just call the entry point
    rts
\end{verbatim}

    \textbf{Load file entry point}
\begin{verbatim}
    ldx #$01               ; First RAM bank used by the editor
    ldy #$ff               ; And last RAM bank
    lda #<fname            ; Pointer to file name (LSB)
    sta $02                ; Store in r0L
    lda #>fname            ; Pointer to file name (MSB)
    sta $03                ; Store in r0H
    lda #fname_end-fname   ; File name length
    sta $04                ; Store in r1L
    jsr $0810              ; Call entry point
    rts

    fname:
        .byt "mytextfile.txt"
    fname_end:
\end{verbatim}

\textbf{Load file with options entry point}
\begin{verbatim}
    ldx #$01               ; First RAM bank used by the editor
    ldy #$ff               ; And last RAM bank
    lda #<fname            ; Pointer to file name (LSB)
    sta $02                ; Store in r0L
    lda #>fname            ; Pointer to file name (MSB)
    sta $03                ; Store in r0H
    lda #fname_end-fname   ; File name length
    sta $04                ; Store in r1L
    lda #$01               ; Auto-indent on, word wrap off
    sta $05                ; Store in r1H
    lda #$04               ; Tab width
    sta $06                ; Store in r2L
    lda #$28               ; Word wrap position
    sta $07                ; Store in r2H
    lda #$08               ; Disk drive device number
    sta $08                ; Store in r3L
    lda #$b1               ; Text white, background light green
    sta $09                ; Store in r3H (screen color)
    lda #$07               ; Text yellow, background black
    sta $0a                ; Store in r4L (header color)
    lda #$00               ; Use default color
    sta $0b                ; Store in r4H (status bar color)
    jsr $0813              ; Call entry point
    rts

fname:
    .byt "mytextfile.txt"
fname_end:
\end{verbatim}

    \subsection{Code samples for the ROM version}
    
    \textbf{Search ROM banks for X16 Edit}

    The X16 Edit ROM bank may be identified by the
    application signature ("X16EDIT") stored at \$FFF0. This
    code sample searches all ROM banks for the signature.
    If found, the ROM bank is returned in A with carry clear; otherwise
    carry is set on return.

\begin{verbatim}
find_me:
    lda $01                ; Store current ROM bank on stack
    pha
    stz $01                ; Prepare searching from ROM bank 0
    ldy #$00

scan:
    lda $fff0,y            ; Signature starts at $fff0
    cmp signature,y
    bne next               ; Signature didn't match, check next ROM bank
    iny                    ; Increase char pointer
    cpy #$07               ; Have we got 7 matching chars? If not, keep looking
    bne scan
    clc                    ; Set C = 0 as indicator X16 Edit was found
    lda $01                ; Load ROM bank into A
    bra exit

next:
    ldy #$00               ; Reset char pointer
    inc $01                ; Select next ROM bank
    lda $01            
    cmp #$20               ; Have we checked all ROM banks?
    bne scan
    sec                    ; Set C = 1 as indicator X16 Edit was not found

exit:
    plx                    ; Restore original ROM bank
    stx $01
    rts

signature: .byt $58,$31,$36,$45,$44,$49,$54     ; = "X16EDIT"
\end{verbatim}

    \textbf{Default entry point}
\begin{verbatim}
    lda $01                ; Store current ROM bank on stack
    pha
    jsr find_me            ; Search ROM banks
    bcs done               ; Exit if X16 Edit wasn't found
    sta $01                ; Set ROM bank
    jsr $c003              ; Call entry point
done:
    pla                    ; Restore original ROM bank
    sta $01
\end{verbatim}

    \textbf{Load file entry point}
\begin{verbatim}
    lda $01                ; Store current ROM bank on stack
    pha
    jsr find_me            ; Search ROM banks
    bcs done               ; Exit if X16 Edit wasn't found
    sta $01                ; Set ROM bank  
    ldx #$01               ; First RAM bank used by the editor
    ldy #$ff               ; And last RAM bank
    lda #<fname            ; Pointer to file name (LSB)
    sta $02                ; Store in r0L
    lda #>fname            ; Pointer to file name (MSB)
    sta $03                ; Store in r0H
    lda #fname_end-fname   ; File name length
    sta $04                ; Store in r1L
    jsr $c003              ; Call entry point
done:
    pla                    ; Restore original ROM bank
    sta $01
    rts

fname:
    .byt "mytextfile.txt"
fname_end:
\end{verbatim}

    \textbf{Load file with options entry point}
\begin{verbatim}
    lda $01                ; Store current ROM bank on stack
    pha
    jsr find_me            ; Search ROM banks
    bcs done               ; Exit if X16 Edit wasn't found
    sta $01                ; Set ROM bank
    
    ldx #$01               ; First RAM bank used by the editor
    ldy #$ff               ; And last RAM bank
    lda #<fname            ; Pointer to file name (LSB)
    sta $02                ; Store in r0L
    lda #>fname            ; Pointer to file name (MSB)
    sta $03                ; Store in r0H
    lda #fname_end-fname   ; File name length
    sta $04                ; Store in r1L
    lda #$01               ; Auto-indent on, word wrap off
    sta $05                ; Store in r1H
    lda #$04               ; Tab width
    sta $06                ; Store in r2L
    lda #$28               ; Word wrap position
    sta $07                ; Store in r2H
    lda #$08               ; Disk drive device number
    sta $08                ; Store in r3L
    lda #$b1               ; Text white, background light green
    sta $09                ; Store in r3H (screen color)
    lda #$07               ; Text yellow, background black
    sta $0a                ; Store in r4L (header color)
    lda #$00               ; Use default color
    sta $0b                ; Store in r4H (status bar color)
    jsr $c006              ; Call entry point
done:
    pla                    ; Restore original ROM bank
    sta $01
    rts
   
fname:
    .byt "mytextfile.txt"
fname_end:
\end{verbatim}

\section{License}
	Copyright 2020--2022, Stefan Jakobsson.

	The X16 Edit program, including this manual, is released under the GNU General Public License version 3 or later.
    The program is free software and comes with ABSOLUTELY NO WARRANTY. You may redistribute and/or modify it under the 
    terms of the GNU General Public License as pub­lished by the Free Software Foundation, either version 3 of the License, 
    or, at your option, any later version. For detailed terms see license file distributed with the program. 
    Also available from \href{https://www.gnu.org/licenses}{www.gnu.org/licenses}.

\end{document}
